\documentclass{article}
\usepackage{graphicx, amsmath, amssymb, geometry, booktabs, url}
\geometry{a4paper, margin=1in}

\title{Comparison of Simulated Annealing and Genetic Algorithm for the Traveling Salesman Problem}
\author{Vandan Amin}
\date{November 19, 2024}

\begin{document}

\maketitle

\section{Introduction}
The Traveling Salesman Problem (TSP) is a classical combinatorial optimization problem that aims to find the shortest possible route visiting a set of cities exactly once and returning to the starting city. Due to its NP-hard nature, heuristic approaches such as Simulated Annealing (SA) and Genetic Algorithm (GA) are often employed to find approximate solutions. This project presents a comparison between SA and GA using different TSP datasets. 


\section{Problem Definition}
The datasets used in this project were taken from TSPLIB \cite{tsplib}. The datasets used in this study include the following:
\begin{itemize}
    \item \texttt{a280.tsp}
    \item \texttt{bayg29.tsp}
    \item \texttt{bays29.tsp}
    \item \texttt{brazil58.tsp}
    \item \texttt{ch130.tsp}
    \item \texttt{br17.atsp}
    \item \texttt{ft53.atsp}
    \item \texttt{kro124p.atsp}
    \item \texttt{p43.atsp}
    \item \texttt{ry48p.atsp}
\end{itemize}
The performance of both algorithms is evaluated in terms of the total distance found and the computation time required.

\section{Environment}
The experiments were conducted on a MacBook Air with the following specifications and tools:
\begin{itemize}
    \item System: MacBook Air (Apple M1, 8GB RAM, macOS)
    \item Software and Tools: Visual Studio Code (IDE), Clang (compiler, version 13.0.0)
    \item Programming Language: C
    \item Compiler Flags: \texttt{-O2} (optimization)
    \item Headers Used: \texttt{stdio.h}, \texttt{stdlib.h}, \texttt{string.h}, \texttt{math.h}, \texttt{time.h}, \texttt{limits.h}
\end{itemize}

\section{Methodology}
\subsection{Simulated Annealing}
Simulated Annealing is a probabilistic technique that approximates the global optimum of a function. In this approach, an initial random solution is generated, and iteratively improved through random swaps of city positions. The acceptance of a worse solution is controlled by a probability function based on the temperature, which gradually decreases following a cooling schedule. Key parameters include:
\begin{itemize}
    \item \textbf{Initial Temperature}: 10000.0
    \item \textbf{Cooling Rate}: 0.995
    \item \textbf{Max Iterations}: 10000
\end{itemize}

\subsection{Genetic Algorithm}
The Genetic Algorithm is a population-based approach inspired by natural selection. An initial random population of candidate solutions is generated and evolved through selection, crossover, and mutation. Key parameters include:
\begin{itemize}
    \item \textbf{Population Size}: 50
    \item \textbf{Mutation Rate}: 0.1
    \item \textbf{Max Iterations}: 10000
\end{itemize}
The GA involves fitness calculation, tournament selection for parents, one-point crossover to generate offspring, and random mutation.

\section{Experimental Results}
The results of applying Simulated Annealing and Genetic Algorithm to the TSP instances are summarized in Table~\ref{tab:results}.

\begin{table}[h!]
    \centering
    \begin{tabular}{lcccc}
        \toprule
        \textbf{Dataset} & \textbf{SA Total Distance} & \textbf{SA Time (s)} & \textbf{GA Total Distance} & \textbf{GA Time (s)} \\
        \midrule
        a280.tsp & 2808 & 0.007 & 29732 & 0.448 \\
        bayg29.tsp & 1922 & 0.003 & 3284 & 0.069 \\
        bays29.tsp & 2474 & 0.001 & 4080 & 0.085 \\
        brazil58.tsp & 31680 & 0.002 & 85559 & 0.129 \\
        ch130.tsp & 14026 & 0.004 & 39841 & 0.237 \\
        br17.atsp & 42 & 0.001 & 54 & 0.063 \\
        ft53.atsp & 10348 & 0.002 & 19734 & 0.121 \\
        kro124p.atsp & 66292 & 0.003 & 158651 & 0.188 \\
        p43.atsp & 5716 & 0.002 & 6210 & 0.107 \\
        ry48p.atsp & 18660 & 0.002 & 37405 & 0.114 \\
        \bottomrule
    \end{tabular}
    \caption{Comparison of Simulated Annealing and Genetic Algorithm on different TSP datasets}
    \label{tab:results}
\end{table}

\section{Analysis}
The experimental results reveal that Simulated Annealing consistently finds shorter routes compared to the Genetic Algorithm for each dataset. Moreover, SA is significantly faster in all cases. The advantage of SA may be attributed to its straightforward iterative improvement mechanism, while GA's reliance on crossover and mutation can introduce randomness that may slow convergence and result in suboptimal solutions.

Genetic Algorithm, while useful in exploring a broader solution space, faces challenges due to the stochastic nature of mutation and crossover, which might lead to local optima without achieving better solutions. On the other hand, Simulated Annealing's gradual cooling mechanism allows for a more focused search, which helps in minimizing the overall travel distance more efficiently.

\section{Conclusion}
This study demonstrates that Simulated Annealing outperforms the Genetic Algorithm in both solution quality and computational efficiency for the tested TSP datasets. Future work may involve tuning the Genetic Algorithm parameters or exploring hybrid approaches that combine the strengths of both algorithms to achieve better performance.

\bibliographystyle{plain}
\begin{thebibliography}{1}

\bibitem{tsplib} Reinelt, G. TSPLIB - A Traveling Salesman Problem Library. \url{http://comopt.ifi.uni-heidelberg.de/software/TSPLIB95/}.

\end{thebibliography}

\end{document}

